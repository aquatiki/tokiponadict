% The grammar section of the Cheat Sheet, rewritten to
% use sitelen pona.
\documentclass{article}
\usepackage{fontspec}
\usepackage{multicol}
\usepackage{titlesec}
\usepackage{changepage}
\usepackage{hyperref}
\usepackage[left=1.0in,top=0.7in,bottom=0.6in]{geometry}
\titlespacing*{\section}{0pt}{0.3cm}{0.0cm}
\newfontface\tpf{linjapona49}
\pagestyle{myheadings}
\markboth{}{A Graphical Grammar of Toki Pona}
\newcommand\tp[1]{{\tpf #1}}

\title{A Graphical Grammar of Toki Pona \\[1ex]{\tp{nasin toki sitelen pi toki+pona}}}
\author{jan Suno en jan Lapi}

\begin{document}
\maketitle
This article explains the rules of Toki Pona grammar
using only the \textit {sitelen pona} hieroglyphic
writing system in examples.  For help with the pronunciation,
definition, and spelling using the Latin alphabet, see
\textit {A Graphic Dictionary of Toki Pona}.
\footnote{Typesetting with LaTeX.  The font is
  \href{http://musilili.net/linja-pona/}{linja pona}.
  Initial explanations are from the
  \href{https://blinry.org/toki-pona-cheat-sheet}{Toki
    Pona Cheat Sheet}}

\newenvironment{ex}{
  \begin{adjustwidth}{0.5cm}{}}{\end{adjustwidth}}

\setlength{\parskip}{0.3em}
\begin{multicols}{2}
\raggedright

\section*{Basic sentences}
The particle \tp{li} separates the subject and the verb:
\begin{ex}
  \tp{soweli li moku} = The cat is eating.

  \tp{jan li lape} = The person is sleeping.
\end{ex}

There is no verb “to be”; the part after \tp{li} can also be
a noun or an adjective.
\begin{ex}
  \tp{kili li moku.} = Fruits are food.

  {\tpf telo li pona}. = Water is good.

  {\tpf telo li moku}. = Water is a drink.
\end{ex}

If the subject is {\tpf mi} or {\tpf sina}, the {\tpf li}
is always omitted:
\begin{ex}
  {\tpf mi moku}. = I eat.

  {\tpf sina pona}. = You are good.
\end{ex}

\section*{Modifying words }
Words can be modified by appending other words.  With just
a single modifer, just put it right after the main word:
\begin{ex}
  {\tpf jan lili} = small human = child

  {\tpf tomo mi} = my house

  {\tpf pilin pona} = to feel good = to be happy
\end{ex}

To modify with a group of words, separate them with
the particle {\tpf pi}. Although it can often be thought
of as meaning “of”, the grammatical function of {\tpf pi} is
actually different: it regroups the following words.
Note the difference:
\begin{ex}
  {\tpf tomo telo nasa} = crazy water house = strange bathroom

  {\tpf tomo pi++telo nasa} = house of the crazy water = pub

  {\tpf jan wawa ala} = no strong people

  {\tpf jan pi++wawa ala} = people of not-strong = weak people
\end{ex}

{\tpf pi} can only be used when followed by multiple words.

\section*{Ambiguity}
You’ll often need to know the context to decide what
things mean.  Many words have multiple or general meanings:
\begin{ex}
  {\tpf soweli} = cat / dog / (any land mammal)
\end{ex}

Many words can play the role of a noun, adjective, or verb:
\begin{ex}
  {\tpf telo} = water / wet / to wash

  {\tpf pona} = good, simple / to fix, to repair
\end{ex}

Nouns have no singular or plural, and no definite or
indefinite article:
\begin{ex}
  {\tpf kili} = a fruit / the fruit / some fruits / the fruits
\end{ex}

There are ways to narrow down \textit{which} object
you are talking about:
\begin{ex}
  {\tpf kili ni li ike}. = \textit{This} apple is bad.

  {\tpf kili mi li ike}. = \textit{My} apple is bad.

  {\tpf kili pi++jelo walo li pona}. = The \textit{light yellow}
  apple is good.

  {\tpf kili li lon supa.  kili ni li pona.} =
  The apple is on the table.  That apple is good.
\end{ex}

\section*{Direct objects}
The particle {\tpf e} separates a direct object from
the rest of the sentence:
\begin{ex}
  {\tpf soweli li moku e telo.} = The cat drinks the water.

  {\tpf mi telo e soweli.} = I’m washing the cat.
\end{ex}

\section*{Negation}
To negate a word, append {\tpf ala}:
\begin{ex}
  {\tpf mi lape ala.} = I'm not sleeping.

  {\tpf jan ala li toki.} = Nobody is talking.
\end{ex}

\section*{Questions}
To ask yes-or-no questions, replace the verb with
“(verb) {\tpf ala} (verb)”:
\begin{ex}
  {\tpf sina ken ala ken lape} = Are you able to sleep?

  {\tpf soweli li wile ala wile moku} = Is the cat hungry?
\end{ex}

Alternatively, append {\tpf anu seme} (``or what”) to
the sentence:
\begin{ex}
  {\tpf sina wile uta e mi anu seme?} = Do you want to kiss me?
\end{ex}

To answer these questions, reply with either “(verb)” or
“(verb) {\tpf ala}”.
\begin{ex}
  {\tpf wile uta} = Yes, I want to kiss you.

  {\tpf wile uta ala} = No, I do not want to kiss you.
\end{ex}

To ask questions that can’t be answered with yes or no,
write a normal sentence and replace the word in question
with {\tpf seme}:
\begin{ex}
  {\tpf sina moku e seme?} = What are you eating?

  {\tpf seme li moku e kili mi?} = Who/what ate my fruit?
\end{ex}

\section*{Providing context}
To provide context for a sentence, prepend another
sentence or expression, followed by {\tpf la}.
This often results in a structures like
“If (part 1), then (part 2)” or “In the context of
(part 1), (part2).”
\begin{ex}
  {\tpf mi lape la ali li pona}. = When I’m asleep,
  everything is okay.

  {\tpf mi la kili li pona}. = To me, fruit is good.
\end{ex}

The \textit{context} of a sentence is not the same
thing as it's \textit{subject}.

\section*{Time and Tense}
Verbs have no tense:
\begin{ex}
  {\tpf mi moku.} = I am eating. / I was eating. / I will be eating.
\end{ex}

Instead, use a {\tpf la}-clause to add a temporal context
to a sentence:
\begin{ex}
{\tpf tenpo ni la mi lape}. = I am sleeping right now.

{\tpf tenpo kama la mi lape}. = I will be sleeping in the future.

{\tpf tenpo pini la mi lape}. = I slept in the past.

{\tpf tenpo suno kama la mi tawa lon pi++tomo esun}. =
Tomorrow I will go to the store.
\end{ex}

\section*{Compound sentences}
Separate multiple subjects in a sentence using {\tpf en}:
\begin{ex}
  {\tpf lape en moku li suli}. = Sleep and food are important.
\end{ex}

To say that the subject does more than one thing,
you can use multiple {\tpf li}-clauses:
\begin{ex}
  {\tpf pipi li moku li pakala}. = The bug eats and destroys.
\end{ex}

If there are several direct objects of the same verb,
you can use multiple {\tpf e}-clauses:
\begin{ex}
  {\tpf mi moku e kili e telo}. = I consume fruit and water.
\end{ex}

\section*{Names}
Names of countries, languages, or people are
treated like adjectives. They are attached to a
noun indicating what class of thing is being named, and
often simplified to Toki Pona’s limited alphabet.
In sitelen pona, the individual letters are then replaced
by a series of signs within a cartouche, where only the
initial sound of each sign is used.  Exactly which words to use
is an artistic choice:
\begin{ex}
  {\tpf [\_kili\_ale\_pilin\_ijo\_lupa\_en]} = Kapile = Gabriele

  {\tpf ma [\_ko\_awen\_nena\_anu\_taso\_awen] li pona lukin}.
  = Kanata = Canada is pretty.

  {\tpf mi toki ala e toki [\_ilo\_ni\_lete\_ilo]}.
  = Inli = I don't speak English.

  {\tpf ma tomo [\_noka\_uta\_jo\_olin\_kon\_akesi] li suli}.
  = Nujoka = New York is big.
\end{ex}

This use of cartouches is why the \textit{sitelen pona} sign for
\textit{name} or \textit{word} is {\tpf nimi}.

Note: with Latin letters, names are capitalized.

\section*{Prepositions}
{\tpf lon, kepeken, tawa}, and {\tpf tan}
can be used as prepositions at the end of a sentence
to modify the verb:
\begin{ex}
  {\tpf mi moku lon tomo.} = I eat in the house.

  {\tpf mi moku kepeken ilo moku.} = I eat using a fork.

  {\tpf sina pona tawa mi.} = You are good for me. = I like you.

  {\tpf sina tawa weka tan seme} = Why are you leaving?
\end{ex}

In proper Toki Pona, prepositions do \textit{not} modify nouns.

\section*{Commands}
Use {\tpf o} and then what you want the person to do:
\begin{ex}
  {\tpf o lukin e ni}! = Look at this!
\end{ex}

To address someone, start a sentence with “(person) {\tpf o},”:
\begin{ex}
  {\tpf jan [\_meli\_esun\_luka\_ilo\_noka] o, sina pona lukin.} = Malin, you are pretty.
\end{ex}

Also use this together with a command, merging the two
{\tpf o}’s:

\begin{ex}
  {\tpf jan [\_sama\_anpa\_moku] o tawa tomo sina.} = Sam, go home.
\end{ex}
\section*{Numbers}
Combine number words to add them up :
\begin{ex}
  {\tpf wan} = 1, {\tpf tu} = 2, {\tpf luka} = 5

  {\tpf tu luka wan } = 11
\end{ex}
\end{multicols}
\end{document}
